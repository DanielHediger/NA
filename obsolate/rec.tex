\subsection{Recycling}
Nahezu jedes Produkt welches wir kaufen hat eine Verpackung. Diese Verpackung muss ebenfalls hergestellt werden und hat somit auch eine Betrachtung bezüglich der Nachhaltigkeit nötig. In unserem Beispiel betrachten wir nur Wein welcher in Glasflaschen abgefüllt werden, welcher auch der Grossteil ausmacht welcher der Endkonsument im Laden kauft. Ausserdem werden sonstige Verpackungsmaterialen wie Karton oder Luftpolsterfolie oder Ähnliches, welche beim Transport eventuell anfallen könnten, vernachlässigt. 
Das Material Glas entsteht beim Schmelzen einer Mischung, die unter anderem Soda, Quarzsand und Kalk enthält. Das zusammenschmelzen geschieht bei etwa 1500 Grad Celsius und benötigt somit sehr viel Energie. Wird bei der Herstellung von Glas zusätzlich rezykliertes Material verwendet, so kann bis zu einem Viertel dieser Energie eingespart werden. Wichtigster Energieträger ist Erdgas bei der Produktion von Glas.\cite{_glas:}
\subsubsection{Schweiz}
Die Schweizer und Schweizerinnen sind bezüglich Recycling von Glas Spitzenreiter, denn es werden circa 95\% des Glases fachgerecht entsorgt und dann wiederverwendet. 
Eine der Gründe für eine solch positive Zahl dürfte sicher sein das es in der Schweiz rund 22`000 Sammelstelle für Glas\cite{_recycling:} gibt, welches nach Glasfarben getrennt ist um das wiederverwenden effizienter zu gestallten. Nachdem das Altglas gesammelt wurde, wird es von Fremdkörper gereinigt und mit einem Glasbrecher in die optimale Form zur Wiederverwendung gebracht. Durch diesen Prozess kann man wieder Energie einsparen. 

Durch das wiederverwerten von Altglas gewinnt man also gleich auf mehreren Ebenen, als erstes schont man die Ressourcen von der Natur, sprich Quarzsand, Soda und Kalk, sowie auch an Energie, welche Hauptsächlich aus dem fossilen Energieträger Erdgas kommt. Glas ist also für das Recycling wie geschaffen, denn heute ist es technisch möglich, eine neue Flasche ohne Qualitätseinbusse aus Altglas herzustellen. Um den geforderten Farbton zu erhalten, wird in der Praxis oft nur etwa 85\% des Altglases pro neue Flasche eingesetzt. Ein Teil des Altglases geht in den Export sowie in die Alternativverwertung und wird unter anderem zu hochwertigem Schaumglas verarbeitet, das im Bau verwendet wird. 
In allen Fällen gilt: Glas bleibt Glas. \cite{_glas:}

\subsubsection{Kalifornien}
In den USA wird der meiste Abfall immer noch verbrannt oder in Deponien verscharrt. Kalifornien hat
aber eines der besten Recycling Netzwerke der Staaten. Dennoch haben nur 64\%  der Kellereien ein
Konzept zur Mülltrennung. 

Beim Recycling von Glas sieht es aber deutlich besser aus. Fast alle Kellereien trennen
wiederverwertbares Glas und fast zwei Drittel trafen Massnahmen um Glasbruch zu reduzieren.

\cite{_2015_cswa_sustainability_report.pdf}

Die Recycling-Quote beim Endverbraucher hingegen ist schlecht. Sie betrug im Jahr 2015 in den ganzen
USA nur 34\%. 