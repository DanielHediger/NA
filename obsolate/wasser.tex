\subsection{Wasser}
Der Wasserverbrauch pro liter Wein liegt bei durchschnittlich 870l Wasser pro Liter Wein (Mekonnen \& Hoekstra, 2011). Der Wasserverbrauch beim Weinanbau ist grundsätzlich nicht so hoch wie bei anderen Gütern. Doch Wasser ist eine essentielle Ressource für das Leben und Wein ein Luxusgut. Deshalb ist es wichtig den Faktor Wasser beim Wein zu untersuchen, wenn man die Nachhaltigkeit von Wein untersucht.
\subsubsection{Schweiz}
Die Schweiz wird oft als das Wasserschloss von Europa bezeichnet. «Viele wichtige Flüsse Europas – Rhein, Rhone, Inn (Donau) und Tessin (Po), Etsch (Adige) – nehmen ihren Ursprung hierzulande. Obschon die Schweiz flächenmässig nur knapp vier Promille am Kontinent ausmacht, befinden sich auf ihrem Boden sechs Prozent der Süsswasservorräte Europas». \cite{_weil}
\begin{figure}[H]
	\centering
	\includegraphics[width=0.9\textwidth]{WV}
	\caption{Wasserverbrauch aufgeteilt nach Sektoren}
\end{figure}
Der Gesamtwasserverbrauch in der Schweiz liegt bei 1960 Mio. $m^3/Jahr$ dieser Verbrauch teilt sich auf die Privathaushalte (2\%), Industrie (17\%) und auf den Agrarsektor (83\%) auf. Die Wein und Bierproduktion braucht hierzulande nur 3\% des Agrarwasserverbrauchs. Für die Weinproduktion muss in der Schweiz nur sehr wenig Wasser eingesetzt werden, da die Niederschläge meistens ausreichen. (Wettstein u. a., 2016)
\begin{figure}[H]
	\centering
	\includegraphics[width=0.9\textwidth]{WVA}
	\caption{Anteile des Wasserverbrauchs im       Agrarsektor}
\end{figure}

Durch die grosse Verfügbarkeit und die geringe Nutzung von Wasser für den Weinbau ist der Wasserverbrauch für Wein in der Schweiz nicht problematisch. Jedoch sind die Wasserverschmutzungen die durch die Weinproduktion entstehen nicht unproblematisch. Zur Wasserverschmutzung tragen vor allem Ausschwemmungen von Pestiziden bei.

\subsubsection{Kalifornien}
\label{sub:wasserverbrauch}

In Kalifornien werden etwa 42'000 $m^3$ Wasser für die Landwirtschaft benötigt. Das entspricht  39 \% des gesamten Wasserverbrauchs. 
In Kalifornien herrscht seit 2011 eine Dürre, die Wasserrationierung nötig machte. Daher regulieren über 90 \% der
Weingüter aktiv ihre Bewässerung. Die grösste Einsparung gibt es durch die Tropfbewässerung. Dabei wird das Wasser direkt den Wurzeln  des Weinstocks zugefügt. Dabei verdunstet weniger Wasser unbenutzt.

Zusätzlich fliesst auch kein Wasser ab und es werden keine Nährstoffe ausgeschwemmt. Das reduziert
den Einsatz von Dünger und die Eutrophierung und Verschmutzung des Grundwassers.

Wo es der Untergrund zulässt, wird auf Trockenfeldbau gesetzt.

\cite{_2015_report_appendix.pdf}


Auch bei der Produktion in den Kellereien wurden Massnahmen getroffen, um das verwendete Wasser zu reduzieren. Hier fällt es vor allem für die Reinigung des Gärtanks an. Dieses Wasser wird nun
aufgefangen und wiederverwendet.

