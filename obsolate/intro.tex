
Die Schweizer legen einen grossen Wert auf naturnahes und nachhaltiges Anbauen von
landwirtschaftlichen Gütern. Es gibt kurze Transportwege, genügend Wasser, die Recycling-Quote ist
hoch und die Konsumenten sind sich ihrer Verantwortung bewusst. Die Arbeiter geniessen einen
Mindestlohn, sind sozialversichert und können auf eine gute Rente hoffen.

In Kalifornien hingegen wohnen Amerikaner. Die sind verschwenderisch und kümmern sich nicht um ihre
Umwelt. Sie werfen alles weg und woher die Rohstoffen kommen, ist ihnen egal. Die Arbeit wird von
illegalen Mexikanern verrichtet, die weder ein geregeltes Einkommen haben noch sich gewerkschaftlich
organisieren dürfen.

Soweit die Vorurteile. Nach dieser oberflächlichen Betrachtung ist es eigentlich klar, welcher Wein
aus ökologischen und sozialen Gründen gekauft werden sollte. Doch hält diese Schlussfolgerung auch
einer Life Cycle Analyse stand? Diese Frage soll dieser Report beantworten.
