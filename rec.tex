\subsection{Recycling}
Nahezu jedes Produkt welches wir kaufen hat eine Verpackung. Diese Verpackung muss ebenfalls hergestellt werden und hat somit auch eine Betrachtung bezüglich der Nachhaltigkeit nötig. In unserem Beispiel betrachten wir nur Wein welcher in Glasflaschen abgefüllt werden, welcher auch der Grossteil ausmacht welcher der Endkonsument im Laden kauft. Ausserdem werden sonstige Verpackungsmaterialen wie Karton oder Luftpolsterfolie oder Ähnliches, welche beim Transport eventuell anfallen könnten, vernachlässigt. 
Das Material Glas entsteht beim Schmelzen einer Mischung, die unter anderem Soda, Quarzsand und Kalk enthält. Das zusammenschmelzen geschieht bei etwa 1500 Grad Celsius und benötigt somit sehr viel Energie. Wird bei der Herstellung von Glas zusätzlich rezykliertes Material verwendet, so kann bis zu einem Viertel dieser Energie eingespart werden. Wichtigster Energieträger ist Erdgas bei der Produktion von Glas
\subsubsection{Schweiz}



\subsubsection{Kalifornien}
In den USA wird der meiste Abfall immer noch verbrannt oder in Deponien verscharrt. Kalifornien hat
aber eines der besten Recycling Netzwerke der Staaten. Dennoch haben nur 64\%  der Kellereien ein
Konzept zur Mülltrennung. 

Beim Recycling von Glas sieht es aber deutlich besser aus. Fast alle Kellereien trennen
wiederverwertbares Glas und fast zwei Drittel trafen Massnahmen um Glasbruch zu reduzieren.

(\glqq{}California Wine Community Sustainability Report Appendix\grqq{} (2015))

Die Recycling-Quote beim Endverbraucher hingegen ist schlecht. Sie betrug im Jahr 2015 in den ganzen
USA nur 34\%. 