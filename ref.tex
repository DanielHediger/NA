\section{Reflexion – Blockwoche Nachhaltigkeit }
In der ersten Woche vom Februar 2017 fand die Blockwoche zum Thema Nachhaltigkeit statt. Die
Blockwoche, die ebenfalls ein ISA Modul ist, wurde auf Englisch durchgeführt. Vorgängig hat jeder
Teilnehmer und Teilnehmerin einen Fussabdruck von WWF erstellen lassen, bei welchen man herauslesen
konnte,  wie nachhaltig man mit dem Planeten Erde umgeht. Des weiteren wurden Texte gelesen um einen
ersten Überblick in das Thema zu kriegen. Die Woche war so gesehen in zwei Teile gegliedert. Der
erste Teil waren die verschiedenen Vorlesungen von Experten und der zweite Teil war die
Gruppenarbeit, in welcher wir ein Produkt genau unter die Lupe nahmen. Diese Ergebnisse
präsentierten wir dann am Freitag im Plenum. Ausserdem wurde ein sogenannter «Final Report» über
dieses Produkt geschrieben. Der Fokus bei diesem Modul lag auf der Nachhaltigkeit von Lebensmittel. 

Die erste «Keynote» Vorlesung war von Brigitte Zogg, welche uns das Thema Nachhaltigkeit bei Coop
näher brachte. Es wurden unter anderem die Programme,  welche Coop unterstützt, vorgestellt, wie
auch die verschiedenen Labels, welche auf Produkten bei Coop zu finden sind. Am Nachmittag ergänzte
Jens Geissel noch zu den Labels und zeigte wie viel Information man über ein Produkt herausfinden
kann innerhalb von 15 Minuten. Ebenfalls am Nachmittag zeigte Julius Gallati seine Analyse zu den
«Fussabdrücken»,  welche wir vorgängig gemacht hatten. 

Am Dienstag wurde hauptsächlich über «GMO» und «Biofuel» diskutiert. Aber es gab eine weitere Vorlesung von Albrecht Ehrensperger welche sich mit «Sustainable Development» und «Food Security» befasste. 
Am Mittwoch ging es weiter mit der «Keynote» Vorlesung von Julius Gallati welche uns vorbereitete auf die kommende Gruppenarbeit. Wie bereits erwähnt war die erste Aufgabe der Gruppe Informationen über ihr Produkt zu sammeln, in unserem Fall Wein, und diese dann zu einer Präsentation am Freitagmorgen zusammen zu tragen. Mittwochnachmittag und Donnerstag wurde hauptsächlich an dem Vortrag gearbeitet und am Freitagmorgen präsentiert. 

Die Woche gestaltete sich in zwei Teile, als Wissen vermittelt zu bekommen und Wissen zu erarbeiten.
Die ersten zwei Tage waren durch den schier endlosen Strom an Informationen sehr anspruchsvoll. Die
Diskussion zwischen Pro- und Contra-Parteien bezüglich «GMO» und «BioFuel» am Dienstagnachmittag war
demnach eine Auflockerung. Auch wenn wir zuerst ein wenig skeptisch waren, so lernten wir viel über
die Problematik der beiden Themen. Bei der Vorlesung von Brigitte Zogg störte uns ein wenig, dass
die Präsentation sich nie kritisch mit den Programmen von Coop auseinandergesetzt hat. Sie war zwar
sehr informativ, hatte aber den Charakter einer Werberveranstaltung.  Wie man Labels kritisch
hinterfragt hat uns dann am Nachmittag Jens Geissel erklärt.

Für die Gruppenarbeiten wurde genügend Zeit und eine angenehme Arbeitsumgebung bereitgestellt,  welche wir sehr schätzten. 

Wir denken, dass diese Woche ein Erfolg war. Wir lernten, dass Nachhaltigkeit sehr viele Aspekte hat
und deshalb fast immer eine komplexe Lösung braucht, um nicht die Probleme zu verlagern. Wir sahen
viele innovative Lösungen, wie zum Beispiel das «Laser-Leveling»,  welches es ermöglicht den Wasserverbrauch massiv zu verringern. Wir als Gruppe lernten natürlich viel über die Umweltaspekte von Wein.
