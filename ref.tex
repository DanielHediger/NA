\section{Reflexion – Blockwoche Nachhaltigkeit }
In der ersten Woche vom Februar 2017 fand die Blockwoche zum Thema Nachhaltigkeit statt. Die Blockwoche, die ebenfalls ein ISA Modul ist, wurde auf Englisch durchgeführt. Vorgängig hat jeder Teilnehmer und Teilnehmerin einen Fussabdruck von WWF erstellen lassen, bei welchen man herauslesen konnte wie nachhaltig man mit dem Planeten Erde umgeht. Des Weitern wurden Texte gelesen um einen ersten Überblick in das Thema zu kriegen. Die Woche war so gesehen in zwei Teile gegliedert. Der erste Teil waren die verschiedenen Vorlesungen von Experten und der zweite Teil war die Gruppenarbeit bei welchen wir ein Produkt genau unter die Lupe nahmen um dann am Freitag unsere Ergebnisse im Plenum zu präsentieren. Ausserdem wurde ein sogenannter «Final Report» über dieses Produkt geschrieben. Generell ging es bei dem Modul über die Nachhaltigkeit über die Nachhaltigkeit von Lebensmittel. 
Die erste «Keynote» Vorlesung war von Brigitte Zogg welche uns das Thema Nachhaltigkeit bei Coop näherbrachte. Es wurden unter anderem die Programme welche Coop unterschützt vorgestellt, wie auch die verschiedenen Labels welche auf Produkten bei Coop zu finden sind. Am Nachmittag ergänzte Jens Geissel noch zu den Labels und zeigte wie viel Information man über ein Produkt herausfinden kann innerhalb von 15 Minuten. Ebenfalls am Nachmittag zeigte Julius Gallati seine Analyse zu den «Fussabdrücken» welche wir vorgängig gemacht haben. 
Am Dienstag wurde hauptsächlich über «GMO» und «Biofuel» diskutiert. Aber es gab eine weitere Vorlesung von Albrecht Ehrensperger welche sich mit «Sustainable Development» und «Food Security» befasste. 
Am Mittwoch ging es weiter mit der «Keynote» Vorlesung von Julius Gallati welche uns vorbereitete auf die kommende Gruppenarbeit. Wie bereits erwähnt war die erste Aufgabe der Gruppe Informationen über ihr Produkt zu sammeln, in unserem Fall Wein, und diese dann zu einer Präsentation am Freitagmorgen zusammen zu tragen. Mittwochnachmittag und Donnerstag wurde hauptsächlich an dem Vortrag gearbeitet und am Freitagmorgen präsentiert. 

Die Woche gestaltete sich in zwei Teile, als Wissen vermittelt zu bekommen und Wissen zu erarbeiten.  Die ersten zwei Tage waren durch den schier endlosen Strom an Informationen sehr anspruchsvoll. Die Diskussion zwischen Pro- und Contra-Parteien bezüglich «GMO» und «BioFuel» am Dienstagnachmittag war demnach eine Auflockerung. Auch wenn wir zuerst ein wenig skeptisch waren, so lernten wir viel über die Problematik der beiden Themen. Bei der Vorlesung von Brigitte Zogg störte uns ein wenig das die Präsentation sich nie kritisch mit den Programmen von Coop auseinandergesetzt hatte. Es war zwar sehr informativ aber auch mehr Werbung für Coop als dass es wirklich kritisch betrachtet wurde. Wir hätten uns mehr gewünscht das man beispielsweise auch Labels kritisch hinterfragt hat, wie dies Jens Geisel am Nachmittag dann «nachgeholt» hat. 
Für die Gruppenarbeiten wurde genügend Zeit bereitgestellt und eine angenehme Arbeitsumgebung welche wir sehr schätzten. 

Wir denken das diese Woche ein Erfolg war, weil wir lernten das Nachhaltigkeit sehr viele Aspekte hat und deshalb fast immer eine komplexe Lösung braucht, um nicht die Probleme zu verlagern. Wir sahen viele Innovative Lösungen wie zum Beispiel, das «Laser-Leveling» welches es ermöglicht den Wasserverbrauch massiv zu verringern. Wir als Gruppe lernten natürlich viel über die Umweltaspekte von Wein.
