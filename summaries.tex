\section{Summaries}
\subsection{COOP – Strategies for Sustainable Food}

Coop is one of the largest retailers in Switzerland. Coop can look back on a History spanning over 150 years. Coop is still a cooperative because some several advantages, for example: for sustainability and long-term thinking. The Strategic concept for sustainability by Coop stand on three columns, which are:\\

-Sustainable assortment performance\\
-Resource efficiency and climate protection\\
-Employees and society\\

All these points are equally important for a company that wants to act sustainably.
In the long history of Coop, they had some milestones about sustainability. For example in 1993, Coop launched Naturaplan, Naturaline and Max Havelaar. 
The Max Havelaar foundation, which is the most famous of Coops launched foundations, is a NGO, which stand for fair-trade, and awards the fair-trade label for sustainable and fair-trade products. The benefits of this organisation is to improve the livelihood of the developing world. You can find these labels on different products such as honey, coffee, tea, bananas, cacao, cotton, pineapples, flowers, mango, orange juice, rice and sugar.
In addition, a big point in the sustainable food strategies by Coop is the thematic about the palm oil. Many areas, which are now palm oil farms, were once rainforests. So many farmers used to burn the rainforest down to increase the amount of the palm oil. Coop is very focused on selling only products from sustainable palm oil cultivation. On this point Coop is working together with the WWF to get on this goal.
For Coop Sustainable palm oil means:
– No uprooting of virgin forests and valuable living space
– Protection of water, soil, air, animals and plants
– Compliance with land use‐ and proprietary‐rights
– No child work, involving small farmers.

Coop is not only trying to sell sustainable products but also to be sustainable. For example, they want to be CO2 Neutral by the year 2023. 
In order to achieve this goal, those used, for example, to transport the goods trucks, which are driven with hydrogen instead of using conventional fuel. 

From 2003 on, they spend each year 10 Million Swiss Francs into the Foundation of Coop Naturaplan‐fund. In addition Coop spend also 16,6 Million Swiss Francs in about 70 projects which have different goals such as innovations, Raising awareness of broad public by broad communication concerning sustainability and also Compensation of CO2‐emissions.

Coop is also committed to sustainability in the social field. This is done for different reasons. One part of the unsold food will be donated to the {}\grqq Swiss table{}\grqq and {}\grqq Tischlein deck dich{}\grqq . Another aspect is that Coop also draws attention to sustainable products, and not least by labels, which have been launched by their own. As mentioned earlier in the text, Coop has launched some own labels.  For Coop, it is also a big concern to be able to infomate its customers and lables are there a good way.

Overall, Coop is a company that is very concerned about sustainability and has been around for a long time. Coop also has ambitious future plans and sustainability is also being improved in various areas.
