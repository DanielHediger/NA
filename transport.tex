In diesem Kapitel wird Lebenszyklus einer Weinflasche untersucht Von dem Anbau bis zum Genuss an einem romantischen Dinner, wird jeder Vorgang einzeln unter die Lupe genommen und sich Gedanken gemacht wie Nachhaltig dieses Luxusgut eigentlich ist. Die Arbeitsschritte wurden Analysiert und die wichtigsten Aspekte in den nachfolgenden Kapiteln untersucht. Es wird versucht aufzuzeigen was für Emission dabei überhaupt entstehen.
\subsection{Transport}
In diesem Abschnitt wird betrachtet welchen Weg eine Flasche von Kalifornien zurücklegt bis bei einem Einzelhändler in Konstanz im Regal entsteht. Hauptaspekt in dieser Betrachtung ist der entstandene Co2 Ausstoss pro Flasche Wein. Grundlage hier dient die \cite{_kohlendioxid-bilanz}. 
\begin{description}
	\item[1 Station]\hfill \\
	Flaschenabfüllung beim Weinbaubetrieb in Acampo, San Joaquin County, 			Kalifornien und sechs Kilometer Transport per LKW (Beladung mit 4.704 			Flaschen) zum Ravenswood-Distribution-Center in Lodi, San Joaquin 				County, Kalifornien
	\item[2 Station]\hfill \\
	Beladung eines 20-PANAMAX-Containers mit insgesamt 12.768 Flaschen 			und 140 Kilometer LKW-Transport des Containers zum Port of Oakland, 			Kalifornien.
	\item[3 Station]\hfill \\
		17.283 Kilometer Seetransport des Containers von Oakland über Panama 			zum Europoort Rotterdam, Niederlande;
	\item[4 Station]\hfill \\
		Umladung des Containers auf ein Rheinschiff und 487 Kilometer Transport 			nach Mainz sowie	
	\item[5 Station]\hfill \\ 
	152 Kilometer LKW-Transport des 20-Fuß-Containers von Mainz bis zum 			Weinhändler nahe Koblenz
\end{description}

Ergebnis einer Studie der Justus-Liebig-Universität Gießen (JLU) in Kooperation mit der San Francisco State University: Der globale Transport einer Flasche kalifornischer Wein über 18.000 Kilometer vom Abfüller in Kalifornien bis zum Einzelhandel gerade einmal 200 Gramm Kohlendioxid zuzurechnen. 200 Gramm Kohlendioxid werden beispielsweise frei, wenn ein privater PKW eine Strecke von nur 1,4 Kilometer zurücklegt.  (Kohlendioxid-Bilanz für kalifornischen Wein überraschend gut \cite{_kohlendioxid-bilanz}. 


CO2 Ausstoss bei der Produktion
Bei der Produktion von Wein fallen durch verschiedene Arbeitsprozesse CO2 Emissionen an. Solche Arbeitsprozesse sind z.B Beschaffung und Verteilung von Dünger mit einem Traktor, Transport der Arbeitskräfte an den Einsatz Ort, Abtransport der Ernte. All dies sind Faktoren die einen unter anderem einen Ausstoss von Kohlendioxid zur Folge haben. In der Schweiz liegt dieser Wert pro ein Kilogramm produzierten Weintrauben zwischen 0.45 kg CO2-eq/Flasche und 0.6 kg CO2-eq/Flasche (ADEME, 2015). Die Variation entsteht durch die verschiedenen Traubensorten welche angebaut werden. In Kalifornien liegt dieser Wert bei 0.3 kg CO2-eq/Flasche bei der Lodi Sorte und bei 0.675 kg CO2-eq/Flasche bei der Napa Sorte.
Anhand dieser Zahlen kann man sagen das der Wein welcher in Kalifornien produziert wird ein bisschen weniger umweltbelastend ist als solcher in der Schweiz. Als Konsument muss sollte man aber beachten das der Wein aus Kalifornien durch seine lange Reise noch um die 0.2 kg CO2-eq/Flasche dazukommen. Dieser Wert ist entgegen der Erwartung eher klein, dies liegt daran das die Weinflaschen hauptsächlich im grossen Verbund auf einem Containerschiff den Grossteil der Strecke zurücklegen.Dazu kommt das nur ein nur ein kleiner Prozentsatz(unter 2\%) des Schweizer Weines exportiert wird und der grossteil direkt in der Schweiz verwertet wird. Die dadurch verursachten Emissionen sind dadurch minimal. 